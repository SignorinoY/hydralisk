% !TEX program = xelatex

\documentclass[answers]{exam}

\usepackage{amsmath}
% \usepackage[complete]{mtpro2}
\usepackage{ctex}
\begin{document}


\begin{questions}
    \question
    贝叶斯学派认为任一未知量 $\theta$ 都可被看作 \fillin[随机变量]。
    \question 若对于二项分布 $b\left(n,\theta\right)$ 中的参数 $\theta$ 无任何了解,我们通常使用 \fillin[均匀分布 $U\left(0,1\right)$] 作为 $\theta$ 的先验分布。
    \question
    对于样本 $X$,参数 $\theta$ 的后验分布为 \fillin[C]。\par
    \begin{oneparchoices}
        \choice $\pi(\theta)$
        \choice $h\left(X,\theta\right)$
        \CorrectChoice $\pi\left(\theta\mid X\right)$
        \choice $p(X\mid\theta)$
    \end{oneparchoices}
    \question
    (高等数理统计 5.16)
    设 $\theta$ 是一批产品的不合格品率,已知它必为 0.1 和 0.2 中之一,且其先验分布为
    \begin{equation*}
        P\left(\theta=0.1\right)=0.7,\quad P\left(\theta=0.2\right)=0.3,
    \end{equation*}
    假如从这批产品中随机取出 8 个进行检查,发现有 2 个是不合格品。求 $\theta$ 的后验分布。
    \question
    (高等数理统计 5.17)
    设 $\theta$ 是一批产品的不合格品率,从中任取 8 个产品进行检验,发现 3 个是不合格品。请在下列先验分布的假设下,分别求 $\theta$ 的后验分布。
    \begin{parts}
        \part
        $\theta\sim U(0,1)$;
        \part
        $\theta\sim\pi(\theta)=\left\{
            \begin{array}{ll}
                2(1-\theta) , & 0<\theta<1,  \\
                0           , & \text{其它}. \\
            \end{array}\right.$
    \end{parts}
    \begin{solution}
        记不合格品的发生次数为 $X$,显然 $x\mid\theta\sim b(8,\theta)$,其中 $0<\theta<1$,所以 $X=3$ 在给定 $\theta$ 时的条件概率为
        \begin{equation*}
            p\left(X=3\mid\theta\right)=\binom{8}{3}\theta^3\left(1-\theta\right)^5.
        \end{equation*}
        \begin{parts}
            \part
            若 $\theta$ 的先验分布为 $\theta\sim U(0,1)$,则 $X$ 和 $\theta$ 的联合分布为
            \begin{equation*}
                h(X=3,\theta)=\binom{8}{3}\theta^{3}\left(1-\theta\right)^{5}.
            \end{equation*}
            $X$ 的边际分布为
            \begin{equation*}
                m(X=3)=\binom{8}{3}\int_{0}^{1}\theta^{3}\left(1-\theta\right)^{5}\mathrm{d}\theta=\binom{8}{3}\frac{\Gamma(4)\Gamma(6)}{\Gamma(10)}.
            \end{equation*}
            因此 $\theta$ 的后验分布为
            \begin{equation*}
                \pi\left(\theta\mid X=3\right)=\frac{h(X=3,\theta)}{m\left(X=3\right)}=\frac{\Gamma(10)}{\Gamma(4)\Gamma(6)}\theta^{3}\left(1-\theta\right)^{5}.
            \end{equation*}
            即 $\theta\mid X=3\sim \text{Be}\left(4,6\right)$。
            \part
            若 $\theta$ 的先验分布为
            \begin{equation*}
                \theta\sim\pi(\theta)=\left\{
                \begin{array}{ll}
                    2(1-\theta) , & 0<\theta<1,  \\
                    0           , & \text{其它}. \\
                \end{array}\right.
            \end{equation*}
            则 $X$ 和 $\theta$ 的联合分布为
            \begin{equation*}
                h(X=3,\theta)=2\binom{8}{3}\theta^{3}\left(1-\theta\right)^{6}.
            \end{equation*}
            $X$ 的边际分布为
            \begin{equation*}
                m(X=3)=2\binom{8}{3}\int_{0}^{1}\theta^{3}\left(1-\theta\right)^{6}\mathrm{d}\theta=2\binom{8}{3}\frac{\Gamma(4)\Gamma(7)}{\Gamma(11)}.
            \end{equation*}
            因此 $\theta$ 的后验分布为
            \begin{equation*}
                \pi\left(\theta\mid X=3\right)=\frac{h(X=3,\theta)}{m\left(X=3\right)}=\frac{\Gamma(11)}{\Gamma(4)\Gamma(7)}\theta^{3}\left(1-\theta\right)^{6}.
            \end{equation*}
            即 $\theta\mid X=3\sim \text{Be}\left(4,7\right)$。

        \end{parts}
    \end{solution}

    \question
    (高等数理统计 5.30)证明:均匀分布 $U\left(0,\theta\right)$ 的参数 $\theta$ 的共轭先验分布是帕雷托(Pareto)分布,其密度函数为
    \begin{equation*}
        \pi(\theta)=\frac{\beta\theta_{0}^{\beta}}{\theta^{\beta+1}},\theta>\theta_{0},
    \end{equation*}
    $\beta,\theta_{0}$ 为两个已知的常数。
    \begin{solution}

    \end{solution}
\end{questions}

\end{document}