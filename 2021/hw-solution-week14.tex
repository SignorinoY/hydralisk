% !TEX program = xelatex

\documentclass[normal,founder,mtpro2,cn]{elegantnote}
    \title{2021年春季学期/数理统计/第十四周/课后作业解答}
    \author{龚梓阳}
    \date{\zhtoday}

\begin{document}
\maketitle
\begin{enumerate}
    \item[1]
        \begin{proof}
            \begin{enumerate}
                \item 犯第一类错误的概率为
                      \begin{equation*}
                          \begin{aligned}
                              \alpha= & P\left\{\bar{x}\in W\mid H_{0}\right\}=P\{\bar{x}\geq 2.6\mid\mu=2\}                                 \\
                              =       & P\left\{\frac{\bar{x}-\mu}{1/\sqrt{n}}\geq\frac{2.6-2}{1/\sqrt{20}}=2.68\right\}=1-\Phi(2.68)=0.0037
                          \end{aligned}
                      \end{equation*}
                      犯第二类错误的概率为
                      \begin{equation*}
                          \begin{aligned}
                              \beta= & P\left\{\bar{x}\notin W\mid H_{1}\right\}=P\{\bar{x}<2.6\mid\mu=3\}                               \\
                              =      & P\left\{\frac{\bar{x}-\mu}{1/\sqrt{n}}<\frac{2.6-3}{1/\sqrt{20}}=-1.79\right\}=\Phi(-1.79)=0.0367
                          \end{aligned}
                      \end{equation*}
                \item 由于
                      \begin{equation*}
                          \beta=P\{\bar{x}<2.6\mid\mu=3\}=P\left\{\frac{\bar{x}-\mu}{1/\sqrt{n}}<\frac{2.6-3}{1/\sqrt{n}}=-0.4\sqrt{n}\right\}=\Phi(-0.4\sqrt{n})\leq 0.01
                      \end{equation*}
                      则
                      \begin{equation*}
                          \Phi(0.4\sqrt{n})\geq 0.99\Rightarrow 0.4\sqrt{n}\geq 2.33\Rightarrow n\geq 33.93
                      \end{equation*}
                      故 $n$ 至少为 34。
                \item
                      \begin{equation*}
                          \begin{aligned}
                              \alpha= & P\{\bar{x}\geq 2.6\mid\mu=2\}=P\left\{\frac{\bar{x}-\mu}{1/\sqrt{n}}\geq\frac{2.6-2}{1/\sqrt{n}}=0.6\sqrt{n}\right\} \\
                              =       & 1-\Phi(0.6\sqrt{n})\rightarrow 0,\quad n\rightarrow\infty
                          \end{aligned}
                      \end{equation*}
                      \begin{equation*}
                          \begin{aligned}
                              \beta= & P\{\bar{x}<2.6 \mid\mu=3\}=P\left\{\frac{\bar{x}-\mu}{1/\sqrt{n}}<\frac{2.6-3}{1/\sqrt{n}}=-0.4\sqrt{n}\right\} \\
                              =      & \Phi(-0.4\sqrt{n})\rightarrow 0,\quad n\rightarrow\infty
                          \end{aligned}
                      \end{equation*}
            \end{enumerate}
        \end{proof}
    \item[2]
        \begin{proof}
            因 $X\sim \text{b}(1,p)$,有 $\sum_{i=1}^{10}x_{i}=10\bar{x}\sim\text{b}(10,p)$。
            则
            \begin{equation*}
                \begin{aligned}
                    \alpha= & P\left\{\bar{x}\in W\mid H_{0}\right\}=P\{\bar{x}\geq 0.5\mid p=0.2\}=P\{10\bar{x}\geq 5\mid p=0.2\} \\
                    =       & \sum_{k=5}^{10}C_{10}^{k}\cdot 0.2^{k}\cdot 0.8^{10-k}=0.0328
                \end{aligned}
            \end{equation*}
            \begin{equation*}
                \begin{aligned}
                    \beta= & P\left\{\bar{x}\notin W\mid H_{1}\right\}=P\{\bar{x}<0.5\mid p=0.4\}=P\{10\bar{x}<5\mid p=0.4\} \\
                    =      & \sum_{k=0}^{4}C_{10}^{k}\cdot 0.4^{k}\cdot 0.6^{10-k}=0.6331
                \end{aligned}
            \end{equation*}
        \end{proof}
    \item[4]
        \begin{proof}
            因均匀分布最大顺序统计量 $x_{(n)}$ 的密度函数为
            \begin{equation*}
                p_{n}(x)=\frac{nx^{n-1}}{\theta^{n}},\quad 0<x<\theta
            \end{equation*}
            则
            \begin{equation*}
                \begin{aligned}
                    \alpha= & P\left\{\bar{x}\in W\mid H_{0}\right\}=P\left\{x_{(n)}\leq 2.5\mid\theta=3\right\}                                                              \\
                    =       & \int_{0}^{2.5}\frac{nx^{n-1}}{3^{n}}\,\mathrm{d}x=\left.\frac{x^{n}}{3^{n}}\right|_{0}^{2.5}=\frac{2.5^{n}}{3^{n}}=\left(\frac{5}{6}\right)^{n}
                \end{aligned}
            \end{equation*}
            要使得 $\alpha\leq 0.05$,即
            \begin{equation*}
                \left(\frac{5}{6}\right)^{n}\leq 0.05\Rightarrow n\geq\frac{\ln 0.05}{\ln(5/6)}=16.43
            \end{equation*}
            故 $n$ 至少为 $17$。
        \end{proof}
    \item[5]
        \begin{proof}
            若检验结果是接受原假设,当原假设为真时,是正确的决策,未犯错误;当原假设不真时,则犯了第二类错误。若检飼结果是拒绝原假设,当原假设为真时,则犯了第一类错误;当原假设不真时,是正确的决策,未犯错误。
        \end{proof}
\end{enumerate}
\end{document}