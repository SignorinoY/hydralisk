% !TEX program = xelatex

\documentclass[normal,founder,mtpro2,cn]{elegantnote}
    \title{2021年春季学期/数理统计/第十一周/课后作业解答}
    \author{龚梓阳}
    \date{\zhtoday}

\begin{document}
\maketitle
\begin{enumerate}
    \item[2]
        \begin{proof}
            由于 $T_{1},T_{2}$ 分别为 $\theta_{1},\theta_{2}$ 的 UMVUE,所以
            \begin{equation*}
                E\left(T_{i}\right)=\theta_{i},\quad i=1,2
            \end{equation*}
            且对于任意满足 $E\left(\varphi\right)=0,\operatorname{Var}\left(\varphi\right)<\infty$ 的 $\varphi$ 有
            \begin{equation*}
                \operatorname{Cov}\left(T_{i},\theta_{i}\right),\quad i=1,2
            \end{equation*}
            因此,
            \begin{equation*}
                E\left(aT_{1}+bT_{2}\right)=a\theta_{1}+b\theta_{2}
            \end{equation*}
            \begin{equation*}
                \operatorname{Cov}\left(aT_{1}+bT_{2},\varphi\right)=a\operatorname{Cov}\left(T_{1},\varphi\right)+b\operatorname{Cov}\left(T_{2},\varphi\right)
            \end{equation*}
            故,由 UMVUE 的判断准则有,$aT_{1}+bT_{2}$ 是 $a\theta_{1}+b\theta_{2}$ 的 UMVUE。
        \end{proof}
    \item[3]
        \begin{proof}
            由于 $T,\hat{g}$ 是 $g\left(\theta\right)$ 的无偏估计,故
            \begin{equation*}
                E\left(T\right)=g\left(\theta\right),\quad E\left(\hat{g}\right)=g\left(\theta\right)
            \end{equation*}
            因此,
            \begin{equation*}
                E\left(T-\hat{g}\right)=0
            \end{equation*}
            由于 $\operatorname{Var}\left(T\right)<\infty,\operatorname{Var}\left(\hat{g}\right)<\infty$,故
            \begin{equation*}
                \operatorname{Var}\left(T-\hat{g}\right)<\infty
            \end{equation*}
            所以,由判断准则有,
            \begin{equation*}
                \operatorname{Cov}\left(T,T-\hat{g}\right)=\operatorname{Var}\left(T\right)-\operatorname{Cov}\left(T,\hat{g}\right)=0
            \end{equation*}
            故,
            \begin{equation*}
                \operatorname{Cov}\left(T,\hat{g}\right)=\operatorname{Var}\left(T\right)\geq 0
            \end{equation*}
        \end{proof}
    \item[5]
        \begin{proof}
            令
            \begin{equation*}
                S_{\theta}=\frac{\partial\ln p\left(x;\theta\right)}{\partial\theta}
            \end{equation*}
            则,
            \begin{equation*}
                \begin{aligned}
                    E\left(S_{\theta}\right)= & \int_{-\infty}^{+\infty}\frac{1}{p\left(x;\theta\right)}\cdot\frac{\partial p\left(x;\theta\right)}{\partial\theta}\cdot p\left(x;\theta\right)\,\mathrm{d}x                            \\
                    =                         & \int_{-\infty}^{+\infty}\frac{\partial}{\partial\theta}p\left(x;\theta\right)\,\mathrm{d}x=\frac{\partial}{\partial\theta}\int_{-\infty}^{+\infty}p\left(x;\theta\right)\,\mathrm{d}x=0
                \end{aligned}
            \end{equation*}
            所以,
            \begin{equation*}
                \frac{\partial}{\partial\theta}E\left(S_{\theta}\right)=0
            \end{equation*}
            同时,
            \begin{equation*}
                \begin{aligned}
                    \frac{\partial E\left(S_{\theta}\right)}{\partial\theta} & =\frac{\partial}{\partial\theta} \int_{-\infty}^{+\infty}S_{\theta}\cdot p\left(x;\theta\right)\,\mathrm{d}x=\int_{-\infty}^{+\infty}\frac{\partial}{\partial\theta}\left[S_{\theta}\cdot p\left(x;\theta\right)\right]\,\mathrm{d}x                                           \\
                    =                                                        & \int_{-\infty}^{\infty}\left[\frac{\partial S_{\theta}}{\partial\theta}\cdot p\left(x;\theta\right)+S_{\theta}\cdot\frac{\partial p\left(x;\theta\right)}{\partial\theta}\right]\,\mathrm{d}x                                                                                  \\
                    =                                                        & \int_{-\infty}^{+\infty}\frac{\partial^{2}\ln p\left(x;\theta\right)}{\partial\theta^{2}}\cdot p\left(x;\theta\right)\,\mathrm{d}x+\int_{-\infty}^{+\infty}\left[\frac{\partial\ln p\left(x;\theta\right)}{\partial\theta}\right]^{2}\cdot p\left(x;\theta\right)\,\mathrm{d}x \\
                    =                                                        & E\left[\frac{\partial^{2}\ln p\left(x;\theta\right)}{\partial\theta^{2}}\right]+E\left(S_{\theta}^{2}\right)                                                                                                                                                                   \\
                    =                                                        & E\left[\frac{\partial^{2}\ln p\left(x;\theta\right)}{\partial\theta^{2}}\right]+I\left(\theta\right)=0
                \end{aligned}
            \end{equation*}
            故,
            \begin{equation*}
                I\left(\theta\right)=-E\left[\frac{\partial^{2}\ln p\left(x;\theta\right)}{\partial\theta^{2}}\right]
            \end{equation*}
        \end{proof}
    \item[6]
        \begin{proof}
            \begin{enumerate}
                \item 样本 $x_{1},x_{2},\ldots,x_{n}$ 的似然函数为
                      \begin{equation*}
                          L\left(\theta\right)=\prod_{i=1}^{n}\theta x_{i}^{\theta-1}
                      \end{equation*}
                      对数似然函数为
                      \begin{equation*}
                          \ln L\left(\theta\right)=n\ln\theta+\left(\theta-1\right)\sum_{i=1}^{n}\ln x_{i}=-n\ln g\left(\theta\right)+\left[\frac{1}{g\left(\theta\right)}-1\right]\sum_{i=1}^{n}\ln x_{i}
                      \end{equation*}
                      令
                      \begin{equation*}
                          \frac{\partial\ln L\left(\theta\right)}{\partial g\left(\theta\right)}=-\frac{n}{g\left(\theta\right)}-\frac{1}{g^{2}\left(\theta\right)}\sum_{i=1}^{n}\ln x_{i}=0
                      \end{equation*}
                      所以,$g\left(\theta\right)$ 的极大似然估计为
                      \begin{equation*}
                          \hat{g}\left(\theta\right)=-\frac{1}{n}\sum_{i=1}^{n}\ln x_{i}
                      \end{equation*}
                \item 令 $Y=-\ln X$,则
                      \begin{equation*}
                          P\left(Y<y\right)=P\left(-\ln X<y\right)=P\left(X>\mathrm{e}^{-y}\right)=\int_{e^{-y}}^{1} \theta x^{\theta-1} \mathrm{~d} x=1-\mathrm{e}^{-\theta y}
                      \end{equation*}
                      因此,
                      \begin{equation*}
                          Y\sim\operatorname{Exp}(\theta),\quad\hat{g}\left(\theta\right)=\frac{1}{n}\sum_{i=1}^{n}Y\sim\operatorname{Ga}\left(n,n\theta\right)
                      \end{equation*}
                      于是,
                      \begin{equation*}
                          E\left(\hat{g}\right)=\frac{n}{n\theta}=\frac{1}{\theta}=g(\theta),\quad\operatorname{Var}\left(\hat{g}\right)=\frac{n}{\left(n\theta\right)^{2}}=\frac{1}{n\theta^{2}}
                      \end{equation*}
                      \begin{equation*}
                          \frac{\partial p\left(x;\theta\right)}{\partial \theta}=\frac{1}{\theta}+\ln x,\quad\frac{\partial^{2}\ln p\left(x;\theta\right)}{\partial\theta^{2}}=-\frac{1}{\theta^{2}}
                      \end{equation*}
                      因此,$\theta$ 的费舍尔信息量为
                      \begin{equation*}
                          I\left(\theta\right)=-E\left[\frac{\partial^{2}}{\partial\theta^{2}}\ln p\left(x;\theta\right)\right]=\frac{1}{\theta^{2}}
                      \end{equation*}
                      故,$g\left(\theta\right)$ 的任一无偏估计的 C-R 下界为
                      \begin{equation*}
                          \frac{\left[g^{\prime}\left(\theta\right)\right]^{2}}{nI\left(\theta\right)}=\frac{1}{n\theta^{2}}
                      \end{equation*}
                      所以,$\hat{g}\left(\theta\right)=-\frac{1}{n}\sum_{i=1}^{n}\ln x_{i}$ 是 $g\left(\theta\right)$ 的有效估计。
            \end{enumerate}
        \end{proof}
    \item[7]
        \begin{proof}
            对数密度函数为
            \begin{equation*}
                \ln p\left(x;\theta\right)=\ln 2+\ln\theta-3\ln x-\frac{\theta}{x^{2}}
            \end{equation*}
            于是,
            \begin{equation*}
                \frac{\partial\ln p\left(x;\theta\right)}{\partial\theta}=\frac{1}{\theta}-\frac{1}{x^{2}},\quad\frac{\partial^{2}\ln p\left(x;\theta\right)}{\partial\theta^{2}}=-\frac{1}{\theta^{2}}
            \end{equation*}
            因此,$\theta$ 的费舍尔信息量为
            \begin{equation*}
                I\left(\theta\right)=-E\left[\frac{\partial^{2}}{\partial\theta^{2}}\ln p\left(x;\theta\right)\right]=\frac{1}{\theta^{2}}
            \end{equation*}
        \end{proof}
    \item[10]
        \begin{proof}
            总体 $\operatorname{Ga}\left(\alpha,\lambda\right)$ 的密度函数为
            \begin{equation*}
                p\left(x;\lambda\right)=\frac{\lambda^{\alpha}}{\Gamma(\alpha)}x^{\alpha-1}\mathrm{e}^{-\lambda x},\quad x>0
            \end{equation*}
            对数密度函数为
            \begin{equation*}
                \ln p\left(x;\lambda\right)=\alpha\ln\lambda-\ln \Gamma\left(\alpha\right)+\left(\alpha-1\right)\ln x-\lambda x \\
            \end{equation*}
            于是,
            \begin{equation*}
                \frac{\partial\ln p\left(x;\lambda\right)}{\partial\lambda}=\frac{\alpha}{\lambda}-x,\quad\frac{\partial^{2}\ln p\left(x;\lambda\right)}{\partial\lambda^{2}}=-\frac{\alpha}{\lambda^{2}}
            \end{equation*}
            因此,$\lambda$ 的费舍尔信息量为
            \begin{equation*}
                I\left(\lambda\right)=-E\left[\frac{\partial^{2}}{\partial\lambda^{2}}\ln p\left(x;\lambda\right)\right]=\frac{\alpha}{\lambda^{2}}
            \end{equation*}
            故,$g\left(\lambda\right)=\frac{1}{\lambda}$ 的任一无偏估计的 C-R 下界为
            \begin{equation*}
                \frac{\left[g^{\prime}\left(\lambda\right)\right]^{2}}{nI\left(\lambda\right)}=\frac{1}{n\alpha\lambda^{2}}
            \end{equation*}
            同时,
            \begin{equation*}
                \frac{\bar{x}}{\alpha}=\frac{1}{n\alpha}\sum_{i=1}^{n}x_{i}\sim\operatorname{Ga}\left(n\alpha,n\alpha\lambda\right)
            \end{equation*}
            因此,
            \begin{equation*}
                E\left(\frac{\bar{x}}{\alpha}\right)=\frac{n\alpha}{n\alpha\lambda}=\frac{1}{\lambda}=g\left(\lambda\right),\quad\operatorname{Var}\left(\frac{\bar{x}}{\alpha}\right)=\frac{n\alpha}{\left(n\alpha\lambda\right)^{2}}=\frac{1}{n\alpha\lambda^{2}}
            \end{equation*}
            故,$\frac{\bar{x}}{\alpha}$ 是 $g(\lambda)=\frac{1}{\lambda}$ 的有效估计,从而也是 UMVUE。
        \end{proof}
    \item[12]
        \begin{proof}
            设 $\varphi\left(x_{1},x_{2},\ldots,x_{n}\right)$ 是 $0$ 的任一无偏估计,则
            \begin{equation*}
                \begin{aligned}
                      & E\left(\varphi\right)=\int_{-\infty}^{+\infty}\ldots\int_{-\infty}^{+\infty}\varphi\cdot\prod_{i=1}^{n}\frac{1}{\sqrt{2\pi}}\exp\left\{-\frac{\left(x_{i}-\mu\right)^{2}}{2}\right\}\,\mathrm{d}x_{1}\ldots\,\mathrm{d}x_{n} \\
                    = & \int_{-\infty}^{+\infty}\ldots\int_{-\infty}^{+\infty}\varphi\cdot(2\pi)^{-\frac{1}{2}}\exp\left\{-\frac{1}{2}\sum_{i=1}^{n}x_{i}^{2}+n\bar{x}\mu-\frac{n\mu^{2}}{2}\right\}\,\mathrm{d}x_{1}\ldots\,\mathrm{d}x_{n}=0
                \end{aligned}
            \end{equation*}
            上式两端对 $\mu$ 求导,有
            \begin{equation*}
                \int_{-\infty}^{+\infty}\ldots\int_{-\infty}^{+\infty}\left(n\bar{x}-n\mu\right)\varphi\cdot(2\pi)^{-\frac{1}{2}}\exp\left\{-\frac{1}{2}\sum_{i=1}^{n}x_{i}^{2}+n\bar{x\mu}-\frac{n\mu^{2}}{2}\right\}\,\mathrm{d}x_{1}\ldots\,\mathrm{d}x_{n}=0
            \end{equation*}
            同时,由于 $E\left(\varphi\right)=0$,有
            \begin{equation*}
                \int_{-\infty}^{+\infty}\ldots\int_{-\infty}^{+\infty}n\bar{x}\varphi\cdot(2\pi)^{-\frac{1}{2}}\exp\left\{-\frac{1}{2}\sum_{i=1}^{n}x_{i}^{2}+n\bar{x\mu}-\frac{n\mu^{2}}{2}\right\}\,\mathrm{d}x_{1}\ldots\,\mathrm{d}x_{n}=0
            \end{equation*}
            上式两端对 $\mu$ 求导,有
            \begin{equation*}
                \begin{aligned}
                      & \int_{-\infty}^{+\infty}\ldots\int_{-\infty}^{+\infty}\left(n\bar{x}\right)^{2}\varphi\cdot(2\pi)^{-\frac{n}{2}}\exp\left\{-\frac{1}{2}\sum_{i=1}^{n}x_{i}^{2}+n\bar{x}\mu-\frac{n\mu^{2}}{2}\right\}\,\mathrm{d}x_{1}\ldots\,\mathrm{d}x_{n} \\
                    - & \int_{-\infty}^{+\infty}\ldots\int_{-\infty}^{\infty}n\bar{x}\cdot n\mu\varphi\cdot(2\pi)^{-\frac{n}{2}}\exp\left\{-\frac{1}{2}\sum_{i=1}^{n}x_{i}^{2}+n\bar{x} \mu-\frac{n\mu^{2}}{2}\right\}\,\mathrm{d}x_{1}\ldots\,\mathrm{d}x_{n}=0
                \end{aligned}
            \end{equation*}
            结合前两式,有
            \begin{equation*}
                E\left(\bar{x}^{2}\varphi\right)=0
            \end{equation*}
            同时,
            \begin{equation*}
                E\left(\bar{x}^{2}\right)=\operatorname{Var}\left(\bar{x}\right)+\left[E\left(\bar{x}\right)\right]^{2}=\frac{1}{n}+\mu^{2},\quad E\left(\bar{x}^{2}-\frac{1}{n}\right)=\mu^{2}
            \end{equation*}

            记
            \begin{equation*}
                T=\bar{x}^{2}-\frac{1}{n}
            \end{equation*}
            则
            \begin{equation*}
                \operatorname{Cov}\left(T,\varphi\right)=E\left(T\varphi\right)-E\left(T\right)E\left(\varphi\right)=0,\quad E\left(T\right)=\mu^{2}
            \end{equation*}
            因此,$T=\bar{x}^{2}-\frac{1}{n}$ 为 $\mu^{2}$ 的 UMVUE。

            由于 $\bar{x}\sim N\left(\mu,1\right)$,有
            \begin{equation*}
                E(\bar{x})=\mu,\quad\operatorname{Var}(\bar{x})=E\left[\left(\bar{x}-\mu\right)^{2}\right]=\frac{1}{n},\quad E\left[\left(\bar{x}-\mu\right)^{3}\right]=0,\quad E\left[\left(\bar{x}-\mu\right)^{4}\right]=\frac{3}{n^{2}}
            \end{equation*}

            则,
            \begin{equation*}
                \begin{aligned}
                    E\left(\bar{x}^{4}\right) & =E\left[\left(\bar{x}-\mu+\mu\right)^{4}\right]                                            \\
                    =                         & E\left[\left(\bar{x}-\mu\right)^{4}\right]+4\mu E\left[\left(\bar{x}-\mu\right)^{3}\right] \\
                                              & +6\mu^{2}E\left[\left(\bar{x}-\mu\right)^{2}\right]+4\mu^{3}E(\bar{x}-\mu)+\mu^{4}         \\
                    =                         & \frac{3}{n^{2}}+\frac{6\mu^{2}}{n}+\mu^{4}
                \end{aligned}
            \end{equation*}
            可得,
            \begin{equation*}
                \begin{aligned}
                    \operatorname{Var}\left(T\right)= & \operatorname{Var}\left(\bar{x}^{2}\right)                                      \\
                    =                                 & E\left(\bar{x}^{4}\right)-\left[E\left(\bar{x}^{2}\right)\right]^{2}            \\
                    =                                 & \frac{3}{n^{2}}+\frac{6\mu^{2}}{n}+\mu^{4}-\left(\frac{1}{n}+\mu^{2}\right)^{2} \\
                    =                                 & \frac{2}{n^{2}}+\frac{4\mu^{2}}{n}
                \end{aligned}
            \end{equation*}

            总体 $N\left(\mu,1\right)$ 的密度函数为
            \begin{equation*}
                p\left(x;\mu\right)=\frac{1}{\sqrt{2\pi}}\exp\left\{-\frac{\left(x-\mu\right)^{2}}{2}\right\}
            \end{equation*}
            对数密度函数为
            \begin{equation*}
                \ln p\left(x;\mu\right)==-\frac{1}{2}\ln\left(2\pi\right)-\frac{\left(x-\mu\right)^{2}}{2}
            \end{equation*}
            于是
            \begin{equation*}
                \frac{\partial}{\partial\mu}\ln p\left(x;\mu\right)=x-\mu
            \end{equation*}
            因此,$\mu$ 的费舍尔信息量为
            \begin{equation*}
                I\left(\mu\right)=E\left[\frac{\partial}{\partial \mu}\ln p\left(x;\mu\right)\right]^{2}=E\left(x-\mu\right)^{2}=1
            \end{equation*}

            故,$g\left(\mu\right)=\mu^{2}$ 的任一无偏估计的 C-R 下界为
            \begin{equation*}
                \frac{\left[g^{\prime}\left(\mu\right)\right]^{2}}{nI\left(\mu\right)}=\frac{\left(2\mu\right)^{2}}{n}=\frac{4\mu^{2}}{n}
            \end{equation*}

            因此,$\operatorname{Var}\left(T\right)=\frac{2}{n^{2}}+\frac{4\mu^{2}}{n}>\frac{4\mu^{2}}{n}$,故 $T=\bar{x}^{2}-\frac{1}{n}$ 不是 $\mu^{2}$ 的有效估计。
        \end{proof}
\end{enumerate}
\end{document}