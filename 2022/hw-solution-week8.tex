\documentclass[normal,cn]{elegantnote}
\title{2022年春季学期/数理统计/第八周/课后作业解答}
\author{龚梓阳}
\date{\zhtoday}

\begin{document}
\maketitle
\begin{enumerate}
    \item[2]
        \begin{proof}
            \begin{enumerate}
                \item
                      样本 $x_{1},x_{2},\ldots,x_{n}$ 的似然函数为
                      \begin{equation*}
                          \begin{aligned}
                              L(\theta)= & \prod_{i=1}^{n}c\theta^{c}x_{i}^{-(c+1)}\mathrm{I}_{x_{i}>\theta}                       \\
                              =          & c^{n}\theta^{nc}\Pi_{i=1}^{n}x_{i}^{-(c+1)}\mathrm{I}_{x_{1},x_{2},\ldots,x_{n}>\theta}
                          \end{aligned}
                      \end{equation*}

                      $c^{n}\theta^{nc}\Pi_{i=1}^{n}x_{i}^{-(c+1)}$ 关于 $\theta$ 单调递增,且由于示性函数的限制,仅当 $x_{1},x_{2},\ldots,x_{n}>\theta$ 时,$L(\theta)>0$。因此,$\theta=\min\left\{x_{1},x_{2},\ldots,x_{n}\right\}=x_{(1)}$ 时,$L(\theta)$ 达到最大。

                      故 $\theta$ 的最大似然估计为 $\hat{\theta}=X_{(1)}$。
                \item
                      样本 $x_{1},x_{2},\ldots,x_{n}$ 的似然函数为
                      \begin{equation*}
                          \begin{aligned}
                              L(\theta,\mu)= & \prod_{i=1}^{n}\frac{1}{\theta}\mathrm{e}^{-\frac{x_{i}-\mu}{\theta}}\mathrm{I}_{x_{i}>\mu}                                      \\
                              =              & \frac{1}{\theta^{n}}\mathrm{e}^{-\frac{1}{\theta}\left(\sum_{i=1}^{n}x_{i}-n\mu\right)}\mathrm{I}_{x_{1},x_{2},\ldots,x_{n}>\mu}
                          \end{aligned}
                      \end{equation*}

                      当 $x_{1},x_{2},\ldots,x_{n}>\mu$ 时,
                      \begin{equation*}
                          \ln L(\theta,\mu)=-n\ln\theta-\frac{1}{\theta}\left(\sum_{i=1}^{n}x_{i}-n\mu\right)
                      \end{equation*}

                      对于给定的 $\theta_{0}$,
                      \begin{equation*}
                          \frac{\partial\ln L(\theta_{0},\mu)}{\partial\mu}=\frac{n}{\theta_{0}}>0
                      \end{equation*}
                      $\ln L(\theta_{0},\mu)$ 关于 $\mu$ 单调递增,且由于示性函数的限制,仅当 $x_{1},x_{2},\ldots,x_{n}>\mu$ 时,$\ln L(\theta_{0},\mu)>0$,因此,$\mu=\min\left\{x_{1},x_{2},\ldots,x_{n}\right\}=x_{(1)}$ 时,$\ln L(\theta_{0},\mu)$ 达到最大。

                      因此,令
                      \begin{equation*}
                          \frac{\partial\ln L(\theta,\hat{\mu})}{\partial\theta}=-\frac{n}{\theta}+\frac{1}{\theta^{2}}\left(\sum_{i=1}^{n}x_{i}-n\hat{\mu}\right)=0
                      \end{equation*}
                      解得 $\theta=\frac{1}{n}\left(\sum_{i=1}^{n}x_{i}-n\hat{\mu}\right)=\bar{x}-\hat{\mu}=\bar{x}-x_{(1)}$。

                      故 $\mu$ 的极大似然估计为 $\hat{\mu}=X_{(1)}$,$\theta$ 的极大似然估计为 $\bar{X}-X_{(1)}$。
                \item
                      样本 $x_{1},x_{2},\ldots,x_{n}$ 的似然函数为
                      \begin{equation*}
                          \begin{aligned}
                              L(\theta)= & \prod_{i=1}^{n}(k\theta)^{-1}\mathrm{I}_{\theta<x_{i}<(k+1)\theta}     \\
                              =          & (k\theta)^{-n}\mathrm{I}_{\theta<x_{1},x_{2},\ldots,x_{n}<(k+1)\theta}
                          \end{aligned}
                      \end{equation*}

                      $(k\theta)^{-n}$ 关于 $\theta$ 单调递减,且由于示性函数的限制,仅当 $\theta<x_{1},x_{2},\ldots,x_{n}<(k+1)\theta$ 时,$L(\theta)>0$。因此,$\theta=\frac{1}{k+1}\max\left\{x_{1},x_{2},\ldots,x_{n}\right\}=\frac{x_{(n)}}{k+1}$ 时,$L(\theta)$ 达到最大。

                      故 $\theta$ 的最大似然估计为 $\hat{\theta}=\frac{X_{(n)}}{k+1}$。
            \end{enumerate}
        \end{proof}
    \item[3]
        \begin{proof}
            \begin{enumerate}
                \item
                      样本 $x_{1},x_{2},\ldots,x_{n}$ 的似然函数为
                      \begin{equation*}
                          L(\theta)=\prod_{i=1}^{n}\frac{1}{2\theta}\mathrm{e}^{-\frac{|x_{i}|}{\theta}}=\frac{1}{2^{n}\theta^{n}}\mathrm{e}^{\frac{1}{\theta}\sum_{i=1}^{n}\left|x_{i}\right|}
                      \end{equation*}
                      其对数似然函数为
                      \begin{equation*}
                          \ln L(\theta)=-n\ln 2-n\ln\theta-\frac{1}{\theta}\sum_{i=1}^{n}|x_{i}|
                      \end{equation*}
                      令
                      \begin{equation*}
                          \frac{\partial\ln L(\theta)}{\partial\theta}=-n\frac{1}{\theta}+\frac{1}{\theta^{2}}\sum_{i=1}^{n}|x_{i}|=0
                      \end{equation*}
                      解得 $\theta=\frac{1}{n}\sum_{i=1}^{n}|x_{i}|$。

                      同时,由于,
                      \begin{equation*}
                          \begin{aligned}
                              \left.\frac{\partial^{2}\ln L(\theta)}{\partial\theta^{2}}\right|_{\theta=\frac{1}{n}\sum_{i=1}^{n}|x_{i}|}= & \left.\left(\frac{n}{\theta^{2}}-\frac{2\sum_{i=1}^{n}|x_{i}|}{\theta^{3}}\right)\right|_{\theta=\frac{1}{n}\sum_{i=1}^{n}|x_{i}|} \\
                              =                                                                                                            & -\frac{n^{3}}{\left(\sum_{i=1}^{n}|x_{i}|\right)^{2}}<0
                          \end{aligned}
                      \end{equation*}
                      故 $\theta$ 的最大似然估计为 $\hat{\theta}=\frac{1}{n} \sum_{i=1}^{n}\left|X_{i}\right|$。
                \item
                      样本 $x_{1},x_{2},\ldots,x_{n}$ 的似然函数为
                      \begin{equation*}
                          L(\theta)=\prod^{n}\mathrm{I}_{\theta-\frac{1}{2}<x_{i}<\theta+\frac{1}{2}}=\mathrm{I}_{\theta-\frac{1}{2}<x_{1},x_{2},\ldots,x_{n}<\theta+\frac{1}{2}}
                      \end{equation*}
                      $L(\theta)$ 仅存在两个取值 0 和 1,且当 $x_{(n)}-\frac{1}{2}<\theta<x_{(1)}+\frac{1}{2}$ 时,有 $L(\theta)=1$。

                      故 $\theta$ 的最大似然估计为 $\hat{\theta}$ 是 $\left(X_{(n)}-\frac{1}{2},X_{(1)}+\frac{1}{2}\right)$ 中任何一个值。
                \item
                      样本 $x_{1},x_{2},\ldots,x_{n}$ 的似然函数为
                      \begin{equation*}
                          \begin{aligned}
                              L\left(\theta_{1},\theta_{2}\right)= & \prod_{i=1}^{n}\frac{1}{\theta_{2}-\theta_{1}}\mathrm{I}_{\theta_{1}<x_{i}<\theta_{2}}                      \\
                              =                                    & \frac{1}{\left(\theta_{2}-\theta_{1}\right)^{n}}\mathrm{I}_{\theta_{1}<x_{1},x_{2},\ldots,x_{n}<\theta_{2}}
                          \end{aligned}
                      \end{equation*}

                      显然 $\theta_{1}$ 越大且 $\theta_{2}$ 越小时, $\frac{1}{\left(\theta_{2}-\theta_{1}\right)^{n}}$ 越大,且由于示性函数的限制,仅当 $\theta_{1}<x_{1},x_{2},\ldots,x_{n}<\theta_{2}$ 时,$L\left(\theta_{1},\theta_{2}\right)>0$。因此,$\theta_{1}=\min\left\{x_{1},x_{2},\ldots,x_{n}\right\}=x_{(1)},\theta_{2}=\max\left\{x_{1},x_{2},\ldots,x_{n}\right\}=x_{(n)}$ 时, $L\left(\theta_{1},\theta_{2}\right)$ 达到最大。

                      故 $\theta_{1}$ 的最大似然估计为 $\hat{\theta}_{1}=X_{(1)}$,$\theta_{2}$ 的最大似然估计为 $\hat{\theta}_{2}=X_{(n)}$。
            \end{enumerate}
        \end{proof}
    \item[5]
        \begin{proof}
            当 $m=2$ 时,$X$ 只能取值 1 或 2,且
            \begin{gather*}
                P\left(X=1\right)=\frac{2p(1-p)}{1-(1-p)^{2}}=\frac{2-2p}{2-p} \\
                P\left(X=2\right)=\frac{p^{2}}{1-(1-p)^{2}}=\frac{p}{2-p}
            \end{gather*}
            因此,
            \begin{equation*}
                P\left(X=x;p\right)=\left(\frac{2-2p}{2-p}\right)^{2-x}\left(\frac{p}{2-p}\right)^{x-1}=\frac{(2-2p)^{2-x}p^{x-1}}{2-p},\quad x=1,2
            \end{equation*}

            样本 $x_{1},x_{2},\ldots,x_{n}$ 的似然函数为
            \begin{equation*}
                L(p)=\prod_{i=1}^{n}\frac{(2-2p)^{2-x_{i}}p^{x_{i}-1}}{2-p}=\frac{(2-2p)^{2n-\sum_{i=1}^{n}x_{i}}p^{\sum_{i=1}^{n}x_{i}-n}}{(2-p)^{n}}
            \end{equation*}
            其对数似然函数为
            \begin{equation*}
                \ln L(p)=\left(2 n-\sum_{i=1}^{n} x_{i}\right) \cdot \ln (2-2 p)+\left(\sum_{i=1}^{n} x_{i}-n\right) \cdot \ln p-n \ln (2-p)
            \end{equation*}

            令
            \begin{equation*}
                \frac{\partial\ln L(p)}{\partial p}=\left(2 n-\sum_{i=1}^{n} x_{i}\right) \cdot \frac{-2}{2-2 p}+\left(\sum_{i=1}^{n} x_{i}-n\right) \cdot \frac{1}{p}-n \cdot \frac{-1}{2-p}=0
            \end{equation*}
            解得
            \begin{equation*}
                p=2-\frac{2n}{\sum_{i=1}^{n}x_{i}}=2-\frac{2}{\bar{x}}
            \end{equation*}

            故 $p$ 的最大似然估计为 $\hat{p}=2-\frac{2}{\bar{X}}$。
        \end{proof}
\end{enumerate}
\end{document}