\documentclass[normal,cn]{elegantnote}
\title{2022年春季学期/数理统计/第十一周/课后作业解答}
\author{龚梓阳}
\date{\zhtoday}

\begin{document}
\maketitle
\begin{enumerate}
    \item[1]
        \begin{proof}
            总体 $X$ 表示一页书上的错别字个数,且
            \begin{equation*}
                X\sim P(\lambda)
            \end{equation*}
            样本为 $x_{1}=3$,有
            \begin{equation*}
                P\left(X_{1}=k\right)=\frac{\lambda^{k}}{k!}\mathrm{e}^{-\lambda},k=0,1,2,\cdots
            \end{equation*}
            则
            \begin{equation*}
                \begin{aligned}
                    P\left(X_{1}=3\right)= & P\left(\lambda=1.5\right)P\left(X_{1}=3\mid\lambda=1.5\right)+P\left(\lambda=1.8\right)P\left(X_{1}=3 \mid \lambda=1.8\right)    \\
                    =                      & 0.45 \times \frac{1.5^{3}}{6} \cdot \mathrm{e}^{-1.5}+0.55 \times \frac{1.8^{3}}{6} \cdot \mathrm{e}^{-1.8}=0.0565+0.0884=0.1449
                \end{aligned}
            \end{equation*}

            故参数 $\lambda$ 的后验分布为
            \begin{gather*}
                P\left(\lambda=1.5\mid X_{1}=3\right)=\frac{P\left(\lambda=1.5\right)P\left(X_{1}=3\mid \lambda=1.5\right)}{P\left(X_{1}=3\right)}=\frac{0.0565}{0.1449}=0.3899\\
                P\left(\lambda=1.8\mid X_{1}=3\right)=\frac{P\left(\lambda=1.8\right)P\left(X_{1}=3\mid \lambda=1.8\right)}{P\left(X_{1}=3\right)}=\frac{0.0884}{0.1449}=0.6101
            \end{gather*}
        \end{proof}
    \item[3]
        \begin{proof}
            \begin{enumerate}
                \item
                      参数 $\theta$ 的先验分布为
                      \begin{equation*}
                          \pi(\theta)=\mathrm{I}_{0<\theta<1}
                      \end{equation*}

                      总体 $X$ 的条件分布为
                      \begin{equation*}
                          P(X=k\mid\theta)=\theta(1-\theta)^{k},\quad k=0,1,2,\ldots
                      \end{equation*}

                      样本 $x_{1},x_{2},\ldots,x_{n}$ 的联合条件分布为
                      \begin{equation*}
                          p\left(x_{1},x_{2},\ldots,x_{n}\mid\theta\right)=\prod_{i=1}^{n}\theta(1-\theta)^{x_{i}}=\theta^{n}(1-\theta)^{\sum_{i=1}^{n}x_{i}}
                      \end{equation*}

                      则 $x_{1},x_{2},\ldots,x_{n}$ 和 $\theta$ 的联合分布为
                      \begin{equation*}
                          h\left(x_{1},x_{2},\ldots,x_{n},\theta\right)=\theta^{n}(1-\theta)^{\sum_{i=1}^{n}x_{i}}\cdot\mathrm{I}_{0<\theta<1}
                      \end{equation*}

                      可得样本 $x_{1},x_{2},\ldots,x_{n}$ 的边际分布为
                      \begin{equation*}
                          m(x_{1},x_{2},\ldots,x_{n})=\int_{-\infty}^{+\infty}\theta^{n}(1-\theta)^{\sum_{i=1}^{n}x_{i}}\cdot\mathrm{I}_{0<\theta<1}\,\mathrm{d}\theta=B\left(n+1,\sum_{i=1}^{n}x_{i}+1\right)
                      \end{equation*}

                      因此 $\theta$ 的后验分布为
                      \begin{equation*}
                          \pi\left(\theta\mid x_{1},x_{2},\ldots,x_{n}\right)=\frac{\theta^{n}(1-\theta)^{\sum_{i=1}^{n}x_{i}}}{B\left(n+1,\sum_{i=1}^{n}x_{i}+1\right)},\quad 0<\theta<1
                      \end{equation*}

                      即,
                      \begin{equation*}
                          \theta\mid x_{1},x_{2},\ldots,x_{n}\sim\text{Be}\left(n+1,\sum_{i=1}^{n}x_{i}+1\right)
                      \end{equation*}
                \item
                      在给定样本 $4,3,1,6$ 的条件下,$\theta$ 的后验分布为
                      \begin{equation*}
                          \theta\mid x_{1},x_{2},x_{3},x_{4}\sim\text{Be}\left(5,15\right)
                      \end{equation*}

                      若采用后验期望估计,则 $\theta$ 的贝叶斯估计为
                      \begin{equation*}
                          \hat{\theta}_{B}=\frac{5}{5+15}=0.25
                      \end{equation*}
            \end{enumerate}
        \end{proof}
    \item[6]
        \begin{proof}
            样本 $x_{1},x_{2},\ldots,x_{n}$ 的联合条件分布为
            \begin{equation*}
                p\left(x_{1},x_{2},\ldots,x_{n}\mid\theta\right)=\prod_{i=1}^{n}\frac{2x_{i}}{\theta^{2}}\mathrm{I}_{0<x_{i}<\theta}=\frac{2^{n}}{\theta^{2n}}\prod_{i=1}^{n}x_{i}\cdot\mathrm{I}_{x_{(n)}<\theta}
            \end{equation*}
            \begin{enumerate}
                \item
                      参数 $\theta$ 的先验分布为
                      \begin{equation*}
                          \pi(\theta)=\mathrm{I}_{0<\theta<1}
                      \end{equation*}

                      则样本 $x_{1},x_{2},\ldots,x_{n}$ 和 $\theta$ 的联合分布为
                      \begin{equation*}
                          h\left(x_{1},x_{2},\ldots,x_{n},\theta\right)=\frac{2^{n}}{\theta^{2n}}\prod_{i=1}^{n}x_{i}\cdot\mathrm{I}_{x_{(n)}<\theta<1}
                      \end{equation*}

                      可得样本 $x_{1},x_{2},\ldots,x_{n}$ 的边际分布为
                      \begin{equation*}
                          m\left(x_{1},x_{2},\ldots,x_{n}\right)=\int_{x_{(n)}}^{1}\frac{2^{n}}{\theta^{2n}}\prod_{i=1}^{n}x_{i}\,\mathrm{d}\theta=\frac{2^{n}}{2n-1}\prod_{i=1}^{n}x_{i}\cdot\left[x_{(n)}^{-(2n-1)}-1\right]
                      \end{equation*}

                      因此 $\theta$ 的后验分布为
                      \begin{equation*}
                          \pi\left(\theta\mid x_{1},x_{2},\ldots,x_{n}\right)=\frac{2n-1}{\theta^{2n}\left[x_{(n)}^{-(2n-1)}-1\right]},\quad x_{(n)}<\theta<1
                      \end{equation*}
                \item
                      参数 $\theta$ 的先验分布为
                      \begin{equation*}
                          \pi(\theta)=3\theta^{2}\cdot\mathrm{I}_{0<\theta<1}
                      \end{equation*}

                      则样本 $x_{1},x_{2},\ldots,x_{n}$ 和 $\theta$ 的联合分布为
                      \begin{equation*}
                          h\left(x_{1},x_{2},\ldots,x_{n},\theta\right)=\frac{3\cdot 2^{n}}{\theta^{2n-2}}\prod_{i=1}^{n}x_{i}\cdot I_{x_{(n)}<\theta<1}=\frac{3\cdot 2^{n}}{\theta^{2n-2}}\prod_{i=1}^{n}x_{i}\cdot I_{x_{(n)}<\theta<1}
                      \end{equation*}

                      可得样本 $x_{1},x_{2},\ldots,x_{n}$ 的边际分布为
                      \begin{equation*}
                          m\left(x_{1},x_{2},\ldots,x_{n}\right)=\int_{x(n)}^{1}\frac{3\cdot 2^{n}}{\theta^{2n-2}}\prod_{i=1}^{n}x_{i}\,\mathrm{d}\theta=\frac{3\cdot 2^{n}}{2n-3}\prod_{i=1}^{n}x_{i}\left[x_{(n)}^{-(2n-3)}-1\right]
                      \end{equation*}

                      因此 $\theta$ 的后验分布为
                      \begin{equation*}
                          \pi\left(\theta\mid x_{1},x_{2},\ldots,x_{n}\right)=\frac{2n-3}{\theta^{2n-2}\left[x_{(n)}^{-(2n-3)}-1\right]},\quad x_{(n)}<\theta<1
                      \end{equation*}
            \end{enumerate}
        \end{proof}
\end{enumerate}
\end{document}