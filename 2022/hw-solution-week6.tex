\documentclass[normal,cn]{elegantnote}
\title{2022年春季学期/数理统计/第六周/课后作业解答}
\author{龚梓阳}
\date{\zhtoday}

\begin{document}
\maketitle
\begin{enumerate}
    \item[3]
        \begin{proof}
            由于 $\hat{\theta}$ 是 $\theta$ 的无偏估计,即 $E(\hat{\theta})=\theta$,因此,
            \begin{equation*}
                E\left[(\hat{\theta})^{2}\right]=\operatorname{Var}(\hat{\theta})+[E(\hat{\theta})]^{2}=\operatorname{Var}(\hat{\theta})+\theta^{2}>\theta^{2}
            \end{equation*}
            故,$(\hat{\theta})^{2}$ 不是 $\theta^{2}$ 的无偏估计。
        \end{proof}
    \item[6]
        \begin{proof}
            由 $X\sim U(0,\theta)$,可知 $x_{(1)},x_{(3)}$ 的密度函数分别为
            \begin{gather*}
                p_{1}(x)=3[1-F(x)]^{2}p(x)=\frac{3(\theta-x)^{2}}{\theta^{3}},\quad 0<x<\theta \\
                p_{3}(x)=3[F(x)]^{2}p(x)=\frac{3x^{2}}{\theta^{3}},\quad 0<x<\theta
            \end{gather*}
            则,
            \begin{gather*}
                E\left(X_{(1)}\right)=\int_{0}^{\theta}x\cdot\frac{3(\theta-x)^{2}}{\theta^{3}}\mathrm{d}x=\left.\frac{3}{\theta^{3}}\left(\theta^{2}\cdot\frac{x^{2}}{2}-2\theta\cdot\frac{x^{3}}{3}+\frac{x^{4}}{4}\right)\right|_{0}^{\theta}=\frac{\theta}{4}\\
                E\left(X_{(1)}^{2}\right)=\int_{0}^{\theta} x^{2}\cdot\frac{3(\theta-x)^{2}}{\theta^{3}}\mathrm{d}x=\left.\frac{3}{\theta^{3}}\left(\theta^{2}\cdot\frac{x^{3}}{3}-2\theta\cdot\frac{x^{4}}{4}+\frac{x^{5}}{5}\right)\right|_{0}^{\theta}=\frac{\theta^{2}}{10}\\
                E\left(X_{(3)}\right)=\int_{0}^{\theta}x\cdot\frac{3x^{2}}{\theta^{3}}dy=\left.\frac{3}{\theta^{3}}\cdot\frac{x^{4}}{4}\right|_{0}^{\theta}=\frac{3\theta}{4} \\
                E\left(X_{(3)}^{2}\right)=\int_{0}^{\theta}x^{2}\cdot\frac{3x^{2}}{\theta^{3}}dy=\left.\frac{3}{\theta^{3}}\cdot\frac{x^{5}}{5}\right|_{0}^{\theta}=\frac{3\theta^{2}}{5}
            \end{gather*}
            因此,
            \begin{equation*}
                E\left(4X_{(1)}\right)=4\cdot\frac{\theta}{4}=\theta,\quad E\left(\frac{4}{3}X_{(3)}\right)=\frac{4}{3}\cdot\frac{3\theta}{4}=\theta
            \end{equation*}
            故,$4X_{(1)}$ 及 $\frac{4}{3}X_{(3)}$ 都是 $\theta$ 的无偏估计;

            同时,
            \begin{gather*}
                \operatorname{Var}\left(4X_{(1)}\right)=16\cdot\left[\frac{\theta^{2}}{10}-\left(\frac{\theta}{4}\right)^{2}\right]=\frac{3\theta^{2}}{5} \\
                \operatorname{Var}\left(\frac{4}{3}X_{(3)}\right)=\frac{16}{9}\cdot\left[\frac{3\theta^{2}}{5}-\left(\frac{3\theta}{4}\right)^{2}\right]=\frac{\theta^{2}}{15}
            \end{gather*}
            故,$\operatorname{Var}\left(4X_{(1)}\right)>\operatorname{Var}\left(\frac{4}{3}X_{(3)}\right)$,即 $\frac{4}{3}X_{(3)}$ 比 $4X_{(1)}$ 更有效。
        \end{proof}
    \item[8]
        \begin{proof}
            因 $T\left(X_{1},\ldots,X_{n}\right)$ 为 $\mu$ 的任一线性无偏估计量,不妨设
            \begin{equation*}
                T\left(X_{1},\ldots,X_{n}\right)=\sum_{i=1}^{n}a_{i}X_{i}
            \end{equation*}
            则,
            \begin{equation*}
                E(T)=\sum_{i=1}^{n}a_{i}E\left(X_{i}\right)=\mu\sum_{i=1}^{n}a_{i}=\mu
            \end{equation*}
            因此 $\sum_{i=1}^{n}a_{i}=1$。

            同时,由于 $X_{1},\ldots,X_{n}$ 相互独立,当 $i\neq j$ 时,有
            \begin{equation*}
                \operatorname{Cov}\left(X_{i},X_{j}\right)=0
            \end{equation*}
            则,
            \begin{equation*}
                \begin{aligned}
                    \operatorname{Cov}(\bar{X},T)= & \operatorname{Cov}\left(\frac{1}{n}\sum_{i=1}^{n}X_{i},\sum_{i=1}^{n}a_{i}X_{i}\right) \\
                    =                              & \sum_{i=1}^{n}\operatorname{Cov}\left(\frac{1}{n}X_{i},a_{i}X_{i}\right)               \\
                    =                              & \sum_{i=1}^{n}\frac{a_{i}}{n}\operatorname{Cov}\left(X_{i},X_{i}\right)                \\
                    =                              & \frac{\sigma^{2}}{n}\sum_{i=1}^{n}a_{i}=\frac{\sigma^{2}}{n}
                \end{aligned}
            \end{equation*}
            因此,
            \begin{equation*}
                \operatorname{Var}(\bar{X})=\frac{1}{n}\operatorname{Var}(X)=\frac{\sigma^{2}}{n}=\operatorname{Cov}(\bar{X},T)
            \end{equation*}
            故 $\bar{X}$ 与 $T$ 的相关系数为
            \begin{equation*}
                \begin{aligned}
                    \operatorname{Corr}(\bar{X},T)= & \frac{\operatorname{Cov}(\bar{X},T)}{\sqrt{\operatorname{Var}(\bar{X})}\sqrt{\operatorname{Var}(T)}} \\
                    =                               & \frac{\operatorname{Var}(\bar{X})}{\sqrt{\operatorname{Var}(\bar{X})}\sqrt{\operatorname{Var}(T)}}   \\
                    =                               & \sqrt{\frac{\operatorname{Var}(\bar{X})}{\operatorname{Var}(T)}}
                \end{aligned}
            \end{equation*}
        \end{proof}
\end{enumerate}
\end{document}