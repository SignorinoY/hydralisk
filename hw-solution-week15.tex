% !TEX program = xelatex

\documentclass[normal,founder,mtpro2,cn]{elegantnote}
    \title{2021年春季学期/数理统计/第十五周/课后作业解答}
    \author{龚梓阳}
    \date{\zhtoday}

\begin{document}
\maketitle
\begin{enumerate}
    \item[3]
        \begin{proof}
            设技术革新后零件质量 $X\sim N\left(\mu,0.05^{2}\right)$,假设
            \begin{equation*}
                \mathrm{H}_{0}:\mu=15\quad\text{ vs }\quad\mathrm{H}_{1}:\mu\neq 15
            \end{equation*}
            由于 $\sigma^{2}$ 已知,选取统计量
            \begin{equation*}
                U=\frac{\bar{x}-\mu}{\sigma/\sqrt{n}}\sim N(0,1)
            \end{equation*}
            给定显著性水平 $\alpha=0.05$,有 $u_{1-\alpha/2}=u_{0.975}=1.96$,因此,双侧拒绝域为
            \begin{equation*}
                W=\{|u|\geq 1.96\}
            \end{equation*}

            且有
            \begin{equation*}
                \bar{x}=14.9,\quad\mu=15,\quad\sigma=0.05,\quad n=6
            \end{equation*}
            则
            \begin{equation*}
                u=\frac{14.9-15}{0.05/\sqrt{6}}=-4.8990\in W
            \end{equation*}
            故拒绝原假设,即不能认为平均质量仍为 15g。
        \end{proof}
    \item[6]
        \begin{proof}
            设这批钢管内直径 $X\sim N\left(\mu,\sigma^{2}\right)$,假设
            \begin{equation*}
                \mathrm{H}_{0}:\mu=100\quad\text{vs}\quad\mathrm{H}_{1}:\mu>100
            \end{equation*}
            \begin{enumerate}
                \item 由于 $\sigma^{2}$ 已知,选取统计量
                      \begin{equation*}
                          U=\frac{\bar{X}-\mu}{\sigma/\sqrt{n}}\sim N(0,1)
                      \end{equation*}
                      给定显著性水平 $\alpha=0.05$,有 $u_{1-\alpha}=u_{0.95}=1.645$,右侧拒绝域为
                      \begin{equation*}
                          W=\{u \geq 1.645\}
                      \end{equation*}
                      且有
                      \begin{equation*}
                          \bar{x}=100.104,\quad\mu=100,\quad \sigma=0.5,\quad n=10
                      \end{equation*}
                      则
                      \begin{equation*}
                          u=\frac{100.104-100}{0.5/\sqrt{10}}=0.6578\notin W
                      \end{equation*}
                      故接受原假设,即不能认为 $\mu>100$。
                \item 由于 $\sigma^{2}$ 未知,选取统计量
                      \begin{equation*}
                          t=\frac{\bar{x}-\mu}{S/\sqrt{n}}\sim t(n-1)
                      \end{equation*}
                      给定显著性水平 $\alpha=0.05$,对于 $n=10$,有 $t_{1-\alpha}(n-1)=t_{0.95}(9)=1.8331$,故右侧拒绝域为
                      \begin{equation*}
                          W=\{t\geq 1.8331\}
                      \end{equation*}
                      且有
                      \begin{equation*}
                          \bar{x}=100.104,\quad\mu=100,\quad s=0.4760,\quad n=10
                      \end{equation*}
                      则
                      \begin{equation*}
                          t=\frac{100.104-100}{0.4760/\sqrt{10}}=0.6910\notin W
                      \end{equation*}
                      故接受原假设,即不能认为 $\mu>100$。
            \end{enumerate}
        \end{proof}
    \item[12]
        \begin{proof}
            设两种型号的计算器充电以后所能使用的时间分别为
            \begin{equation*}
                X\sim N\left(\mu_{1},\sigma_{1}^{2}\right),\quad Y\sim N\left(\mu_{2},\sigma_{2}^{2}\right),\quad\sigma_{1}^{2}=\sigma_{2}^{2}
            \end{equation*}
            假设
            \begin{equation*}
                \mathrm{H}_{0}:\mu_{1}=\mu_{2}\quad\text{vs}\quad \mathrm{H}_{1}:\mu_{1}>\mu_{2}
            \end{equation*}
            由于 $\sigma_{1}^{2},\sigma_{2}^{2}$ 未知,但 $\sigma_{1}^{2}=\sigma_{2}^{2}$。选取统计量
            \begin{equation*}
                T=\frac{\bar{x}-\bar{y}}{s_{w}\sqrt{\frac{1}{n_{1}}+\frac{1}{n_{2}}}}\sim t\left(n_{1}+n_{2}-2\right)
            \end{equation*}
            给定显著性水平 $\alpha=0.01$,对于 $n_{1}=11,n_{2}=12$,有 $t_{1-\alpha}\left(n_{1}+n_{2}-2\right)=t_{0.99}(21)=2.5176$,右侧拒绝域为
            \begin{equation*}
                W=\{t \geq 2.5176\}
            \end{equation*}
            且有
            \begin{gather*}
                \bar{x}=5.5,\quad\bar{y}=4.3667,\quad s_{x}=0.5235, \quad s_{y}=0.4677,\quad n_{1}=11,\quad n_{2}=12 \\
                s_{w}=\sqrt{\frac{\left(n_{1}-1\right)s_{x}^{2}+\left(n_{2}-1\right)s_{y}^{2}}{n_{1}+n_{2}-2}}=\sqrt{\frac{10\times 0.5235^{2}+11\times 0.4677^{2}}{21}}=0.4951
            \end{gather*}
            则
            \begin{equation*}
                t=\frac{5.5-4.3667}{0.4951\times\sqrt{\frac{1}{11}+\frac{1}{12}}}=5.4844\in W
            \end{equation*}
            故拒绝原假设,即可以认为型号 A 的计算器平均使用时间明显比型号 B 来得长。
        \end{proof}
    \item[13]
        \begin{proof}
            设东、西两支矿脉的含锌量分别为
            \begin{equation*}
                X_{1}\sim N\left(\mu_{1},\sigma_{1}^{2}\right),\quad X_{2}\sim N\left(\mu_{2},\sigma_{2}^{2}\right),\quad\sigma_{1}^{2}=\sigma_{2}^{2}
            \end{equation*}
            假设
            \begin{equation*}
                \mathrm{H}_{0}:\mu_{1}=\mu_{2}\quad\text{vs}\quad\mathrm{H}_{1}:\mu_{1}\neq\mu_{2}
            \end{equation*}
            由于 $\sigma_{1}^{2},\sigma_{2}^{2}$ 未知,但 $\sigma_{1}^{2}=\sigma_{2}^{2}$,选取统计量
            \begin{equation*}
                T=\frac{\bar{x}_{1}-\bar{x}_{2}}{s_{w}\sqrt{\frac{1}{n_{1}}+\frac{1}{n_{2}}}}\sim t\left(n_{1}+n_{2}-2\right)
            \end{equation*}
            给定显著性水平 $\alpha=0.05$,对于 $n_{1}=9,n_{2}=8$,有$t_{1-\alpha/2}\left(n_{1}+n_{2}-2\right)=t_{0.975}(15)=2.1314$,双侧拒绝域为
            \begin{equation*}
                W=\{|t|\geq 2.1314\}
            \end{equation*}
            且有
            \begin{gather*}
                \bar{x}_{1}=0.230,\quad s_{1}^{2}=0.1337,\quad\bar{x}_{2}=0.269,\quad s_{2}^{2}=0.1736,\quad n_{1}=9,\quad n_{2}=8\\
                s_{w}=\sqrt{\frac{\left(n_{1}-1\right)s_{1}^{2}+\left(n_{2}-1\right)s_{2}^{2}}{n_{1}+n_{2}-2}}=\sqrt{\frac{8 \times 0.1337+7 \times 0.1736}{15}}=0.3903
            \end{gather*}
            则
            \begin{equation*}
                t=\frac{0.230-0.269}{0.3903\times\sqrt{\frac{1}{9}+\frac{1}{8}}}=-0.2056\notin W
            \end{equation*}
            故接受原假设,即可以认为东、西两支矿脉含锌量的平均值是一样的。
        \end{proof}
    \item[18]
        \begin{proof}
            设两个化验室测定的含气量数据之差为
            \begin{equation*}
                D=X-Y\sim N\left(\mu_{d},\sigma_{d}^{2}\right)
            \end{equation*}
            假设
            \begin{equation*}
                \mathrm{H}_{0}:\mu_{d}=0\quad\text{vs}\quad\mathrm{H}_{1}:\mu_{d}\neq 0
            \end{equation*}
            由于 $\sigma_{d}^{2}$ 未知,选取统计量
            \begin{equation*}
                T=\frac{\bar{D}}{S_{d}/\sqrt{n}}\sim t(n-1)
            \end{equation*}
            给定显著水平 $\alpha=0.01$,对于 $n=7$,有 $t_{1-\alpha/2}(n-1)=t_{0.995}(6)=3.7074$,双侧拒绝域为
            \begin{equation*}
                W=\{|t| \geq 3.7074\}
            \end{equation*}
            且有
            \begin{equation*}
                \bar{d}=-0.0257,\quad s_{d}=0.0922,n=7
            \end{equation*}
            则
            \begin{equation*}
                t=\frac{-0.0257}{0.0922/\sqrt{7}}=-0.7375\notin W
            \end{equation*}
            故接受原假设,可以认为两化验室测定结果之间没有显著差异。
        \end{proof}
    \item[25]
        \begin{proof}
            设两台机器生产金属部件质量分别为
            \begin{equation*}
                X\sim N\left(\mu_{1},\sigma_{1}^{2}\right),\quad Y\sim N\left(\mu_{2},\sigma_{2}^{2}\right)
            \end{equation*}
            假设
            \begin{equation*}
                \mathrm{H}_{0}:\sigma_{1}^{2}=\sigma_{2}^{2}\quad\text{vs}\quad\mathrm{H}_{1}:\sigma_{1}^{2}>\sigma_{2}^{2}
            \end{equation*}
            选取统计量
            \begin{equation*}
                F=\frac{S_{1}^{2}}{S_{2}^{2}}\sim F(m-1,n-1)
            \end{equation*}
            给定显著性水平 $\alpha=0.05$,对于 $m=14,n=12$,有 $F_{1-\alpha}(m-1,n-1)=F_{0.95}(13,11)=2.7614$,右侧拒绝域为
            \begin{equation*}
                W=\{F\geq 2.7614\}
            \end{equation*}
            且有 $s_{1}^{2}=15.46,\quad s_{2}^{2}=9.66,\quad m=14,\quad n=12$
            则
            \begin{equation*}
                F=\frac{15.46}{9.66}=1.6004\notin W
            \end{equation*}
            故接受原假设,即可以认为 $\sigma_{1}^{2}=\sigma_{2}^{2}$。
        \end{proof}
\end{enumerate}
\end{document}